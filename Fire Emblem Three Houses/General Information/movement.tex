\section*{Movement}
\addcontentsline{toc}{section}{Movement}

\subsection*{Monastary Movement}
\addcontentsline{toc}{subsection}{Monastary Movement}
\begin{itemize}
\item The first thing you should do in every monastery segment is to press + to zoom in the camera
\item If you can, hold the right analog stick to point the camera down whenever possible. A claw grip is recommended to be able to hold the right analog stick while holding the B button to run and having a finger on A to be ready to interact with NPCs.
\item The best way to navigate is through utilizing the minimap in the upper-right corner, since ideally you should be staring at the ground the whole time.
\end{itemize}

\subsection*{Battle Movement}
\addcontentsline{toc}{subsection}{Battle Movement}
\begin{itemize}
\item Always point the camera as high as possible with the right analog stick, to reduce lag. The lag reduction is most apparent in fog of war / bigger maps.
\item Hold Y as much as possible for faster cursor movement
\item The d-pad provides faster cursor movement than the left analog stick.
\item Avoid zooming out the camera if you can, because it causes more lag. Chapter 7 is the main exception \item primarily because it's an important chapter to be able to see what's going on at a high level and improvise appropriately.
\item Always attack enemies by selecting them before moving, as opposed to moving your unit first then selecting 'Attack'. The latter requires a minimum of 4 inputs, the former requires 2 inputs. Get used to changing your weapons with X/Y and combat artes with L/R, since that's needed for the former method of attacking.
\item With mounted units, try to avoid being prompted to canto. Every prompt for canto causes the mounted class to do a silly "recoil" animation that costs over a second every time. For this reason, you'll see runners sometimes dismount before an action, just to avoid this. Although extremely situational, you can also draw a suboptimal path and intentionally cause the Canto Bug, so you aren't prompted to canto anymore (See "Canto Bug")
\end{itemize}

\subsection*{L/R Swapping}
\addcontentsline{toc}{subsection}{L/R Swapping}
\begin{itemize}
\item During preps or an actual battle, knowing how the L and R buttons work is critical in moving quickly.
\item The general queue for unit ordering goes from the top row to the bottom row, from left to right. The example grid on the right visualizes this.
\item Pressing R on a unit will jump the cursor to the next UNUSED unit in the queue. For example, if unit 3 already moved, then pressing R on unit 2 will jump the cursor to unit 4. The L button does the same but backwards.
\item Pressing R on an empty tile will jump the cursor to the first UNUSED unit in the queue. For example, if unit 1 and 2 already moved, then pressing R on an empty tile will jump the cursor to unit 3.
\item Pressing L on an empty tile will jump the cursor to the last UNUSED unit in the queue. For example, if unit 6 and 5 already moved, then pressing L on an empty tile will jump the cursor to unit 4.
\end{itemize}


\subsection*{Unit Targetting}
\addcontentsline{toc}{subsection}{Unit Targeting}
\begin{itemize}
\item Unit targetting follows the same pattern as the L/R Swapping pattern:
\item Press Right ($\rightarrow$) to select the next unit to the right in the same row, or the left-most unit in the next row below
\item Press Left ($\leftarrow$) to select the next unit to the left in the same row, or the right-most unit in the next row above
\item If possible, support gambits will always initially target Byleth
\end{itemize}